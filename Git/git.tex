Git is a distributed version control system (VCS) that allows multiple people to work on a project 
simultaneously without interfering with each other's work. 
It tracks changes in files and directories over time,
enabling you to manage and organize your codebase effectively. Below is an overview of Git, its key concepts, and how it works.

%............................ Setting Your Global Username and Email.....................................>>>
Set your global username:

git config --global user.name "Your Name"
% Set your global email:


git config --global user.email "your.email@example.com"
% Setting Username and Email for a Specific Repository

git config --global --list
% View local settings (for the current repository):

git config --list

%...................................Basic Git Commands..................................................>>>

1)git init
% Initializes a new Git repository in the current directory.
git init

2)git clone
% Copies an existing Git repository (usually from a remote server) to your local machine.
git clone <repository_url>

3)git add
% Adds files to the staging area, preparing them to be committed.
git add <file_name>
git add .   # Adds all changes

4)git commit
% Records changes in the repository with a descriptive message.
git commit -m "Your commit message"

5)git status
% Shows the status of changes as untracked, modified, or staged.
git status

6)git push
% Sends your committed changes to a remote repository (e.g., GitHub).
git push origin <branch_name>

7)git pull
% Fetches and integrates changes from a remote repository into your local branch.
git pull origin <branch_name>


8)git branch
% Lists all branches in your repository or creates/deletes branches.
git branch          # List branches
git branch <name>   # Create a new branch
git branch -d <name>  # Delete a branch

9)git checkout
% Switches to a different branch or restores files to a specific state.
git checkout <branch_name>  # Switch to another branch
git checkout -- <file_name>  # Discard changes in a file

10)git merge
% Combines changes from one branch into another branch.
git merge <branch_name>

11)git log
% Shows the commit history for the repository.
git log

12)git diff
% Shows the changes between commits, branches, or your working directory and the staging area.
git diff

13)git remote
% Manages the set of repositories ("remotes") whose branches you track.
git remote -v         # List remotes
git remote add origin <url>  # Add a remote repository

% ..........................................Advanced Git Commands .....................................>>>>

1)git stash
% Temporarily stores changes that are not ready to be committed, allowing you to work on something else.
git stash
git stash pop         # Apply the most recent stash and remove it from the stash list
git stash list        # List all stashed changes

2)git rebase
% Reapplies commits on top of another base tip. This is used to maintain a clean commit history.
git rebase <branch_name>

3)git reset
% Undoes changes by moving the HEAD to a previous commit. Can be used to unstage files or undo commits.
git reset <commit_hash>     # Reset to a specific commit
git reset --soft HEAD~1     # Undo the last commit, keep changes staged
git reset --hard HEAD~1     # Undo the last commit and discard changes

4)git cherry-pick
% Applies changes from specific commits from another branch.
git cherry-pick <commit_hash>

5)git fetch
% Downloads changes from a remote repository, but does not integrate them into your working branch.
git fetch origin

6)git tag
% Marks specific points in the commit history as important, usually for releases.
git tag <tag_name>
git push origin <tag_name>   # Push a tag to a remote

7)git blame
% Shows what revision and author last modified each line of a file.
git blame <file_name>

8)git show
% Shows various types of objects (commits, tags, etc.).
git show <commit_hash>

9)git revert
% Creates a new commit that undoes the changes from a previous commit.
git revert <commit_hash>

10)git clean
% Removes untracked files from your working directory.
git clean -f          # Remove untracked files
git clean -fd         # Remove untracked files and directories

% >>>>>>>>>>>>>>>>>>>>>>>>>>>>>>>>>>>>>>>>>>>>>>>>>>>>>>>>>>>>>>>>>>>>>>>>>>>>>>>>>>>>>>>>>>>>>>>>>>>>>>>>>>>>>>>>>>>>>>>>>>>>>>>>>